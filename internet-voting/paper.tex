%%%%%%%%%%%%%%%%%%%%%%%%%%%%%%%%%%%%%%%%%
% Thin Sectioned Essay
% LaTeX Template
% Version 1.0 (3/8/13)
%
% This template has been downloaded from:
% http://www.LaTeXTemplates.com
%
% Original Author:
% Nicolas Diaz (nsdiaz@uc.cl) with extensive modifications by:
% Vel (vel@latextemplates.com)
%
% License:
% CC BY-NC-SA 3.0 (http://creativecommons.org/licenses/by-nc-sa/3.0/)
%
%%%%%%%%%%%%%%%%%%%%%%%%%%%%%%%%%%%%%%%%%

%-------------------------------------------------------------------------------
%	PACKAGES AND OTHER DOCUMENT CONFIGURATIONS
%-------------------------------------------------------------------------------

\documentclass[a4paper, 11pt]{article} % Font size (can be 10pt, 11pt or 12pt) and paper size (remove a4paper for US letter paper)

\usepackage[protrusion=true,expansion=true]{microtype} % Better typography
\usepackage{graphicx} % Required for including pictures
\usepackage{wrapfig} % Allows in-line images
\usepackage{hyperref} % References
\usepackage[none]{hyphenat} % Don't ever hypenate.
\usepackage{mathpazo} % Use the Palatino font
\usepackage{csquotes} % Block Quotes
\usepackage[T1]{fontenc} % Required for accented characters
\linespread{1.05} % Change line spacing here, Palatino benefits from a slight increase by default

\makeatletter
\renewcommand\@biblabel[1]{\textbf{#1.}} % Change the square brackets for each bibliography item from '[1]' to '1.'
\renewcommand{\@listI}{\itemsep=0pt} % Reduce the space between items in the itemize and enumerate environments and the bibliography

\renewcommand{\maketitle}{ % Customize the title - do not edit title and author name here, see the TITLE block below
\begin{flushright} % Right align
{\LARGE\@title} % Increase the font size of the title

\vspace{50pt} % Some vertical space between the title and author name

{\large\@author} % Author name
\\\@date % Date

\vspace{40pt} % Some vertical space between the author block and abstract
\end{flushright}
}

%-------------------------------------------------------------------------------
%	TITLE
%-------------------------------------------------------------------------------

\title{
    \textbf{Internet Voting}\\ % Title
    Navigating the tangle of concerns.
} % Subtitle

\author{\textsc{Hobden, Andrew} % Author
\\{\textit{University of Victoria}}} % Institution

\date{\today} % Date

%-------------------------------------------------------------------------------

\begin{document}

\maketitle % Print the title section

%-------------------------------------------------------------------------------
%	ABSTRACT AND KEYWORDS
%-------------------------------------------------------------------------------

%\renewcommand{\abstractname}{Summary} % Uncomment to change the name of the abstract to something else

\begin{abstract}
    How well does the Estonia voting system as described by Halderman in the video on Connex address the concerns and recommendations expressed in the BC report on internet voting?  Point out specific features or short-comings if you can find any. Would you want votes to be conducted this way in Canada? Why or why not?
\end{abstract}

\hspace*{3,6mm}\textit{Keywords:} Internet Voting, Operation Security % Keywords

\vspace{30pt} % Some vertical space between the abstract and first section

%-------------------------------------------------------------------------------
%	ESSAY BODY
%-------------------------------------------------------------------------------

\section*{Background}

In two enlightening videos \cite{halderman-estonia-presentation} \cite{halderman-estonia-video}, Alex Halderman offered viewers an overview of Estonia's internet voting system, as well as an inside look at how attackers may be able to compromise such a system. Such compromises need not be obvious but may still grossly affect the outcome of an election.

In BC, an independent panel produced a report \cite{internet-voting-report} about internet voting, ultimately reccommending against adopting universal internet voting.

%-------------------------------------------------------------------------------

\section*{Reviewing the Principles}

The report \cite{internet-voting-report} outlined several principles to evaluate potential systems against. Below we can break down how Estonia's system worked in relation to these specifics:

\begin{description}
    \item[Accessibility] With the Estonian system, the voting client is available for Mac, Windows, and Linux, lending it strong accessibility. There was no research on the system's ability to handle attacks like a DDOS or 0-day vulnerability. Barring some sort of high-capacity attack it's unlikely the system would be inaccessible for the entire week of internet voting. This system satisfactorily addresses the panel's principle.
    \item[Ballot Anonimity] The system featured a two stage process which involved two seperate air-gapped machines. One machine had access to the signed, encrypted ballots, and could validate, then strip the signing off the ballots. The other machine could take the anonymous ballots and decrypt them, ensuring that by the time the vote is counted, the identity has been removed. It's necessary for there to be some trust (and perhaps validation) that the respective machines are adequately secured. This system satisfies this principle.
    \item[Individual and Independent Verifiability] This system provides voters with a reasonable way to review their vote was as they submitted. This verification system utilizes a different device, a smartphone. There is a reasonable limit on the number of times someone could review their vote, and the system allows for resubmission to prevent cohercion. This principle is moderately satisfied.
    \item[Non-reliance on Trustworthiness of the Voter’s Device(s)] This system depends heavily on the users device (and possibly another if they verify their vote) and does not satisfy this principle.
    \item[One Vote Per Voter] The Estonian system only counts the latest vote per voter and uses the country's key infrastructure to ensure the valid identity of each vote. This system satisfies the principle.
    \item[Only Count Votes from Eligible Voters] The Estonian identity infrastructure, with key pairs for citizens, provides this principle in a very strong way.
    \item[Process Validation and Transparency] This system's server software was open sourced, as well as an extensive set of recordings of their procedures. Unfortunately the client source was not provided. This principle was partially satisfied.
    \item[Service Availability] Since Estonia still supports paper ballots, there is an adequate fallback if a disaster strikes the internet voting platform. This adequately satisfies this principle.
    \item[Voter Authentication and Authorization] As previously discussed, the Estonian identity infrastructure accomplishes this principle very well.
\end{description}

Overall, the Estonian system adequately satisfies the majority of principles set out by the report. Areas of concern include the device(s) used for voting, untrustworthiness of the client software, and the accessibility alternative systems if a state-level attacker were to compromise the online voting system.

%-------------------------------------------------------------------------------

\section*{Challenges in Internet Voting}

The report produced a set challenges in internet voting that can be used to evaluate the Estonian system.

\subsection*{Security}

The report detailed three key areas of security to focus on.

\begin{description}
    \item[In Transit] Votes are encrypted and signed with appropriately secured keys. They are then transported appropriately over TLS/SSL streams to the ballot collection server. It is not clear what, if any, checks the voting software does to verify the legitmacy of the ballot collection server. If somehow the SSL traffic was stripped of it's encryption it's possible attackers could identify those who did vote online.
    \item[At the Voting Device] The voting device, the voter's home computer, is quite unsecure. As the report details, problems with malware and virusii are entirely possible and perhaps to be expected. Worse yet, many electors have little working knowledge of computer security and may be fooled into fake voting sites which may skim their credentials. In Estonia's case these dangers are prevelant and certainly cause for concern.
    \item[At the Election Server] The Estonian system breaks the election server into two seperate components air-gapped from one another. The first, the ballot collection server, is only able to connect voters to ballots, and not identify choice of the ballot. The second is an offline server which counts the ballots, and it is provided ballots stripped of their identifying signatures. This system makes many good choices. The offline tallying server is not susceptible to remote intrusion attacks (since it is not connected) and the publicly available video allows for a reasonable assurance against insider threats.
\end{description}

Additionally was noted in the video \cite{halderman-estonia-presentation}, the operational security of the information technology team seemed to be weak. They were noted using personal, unclean laptops and unwiped USB keys for various tasks which would be considered mission critical. Furthermore it was noted several components of their toolkit were obtained in an insecure manner, opening the door for even more attacks.

\section*{Conclusion}
Lorem Ipstuff.


%-------------------------------------------------------------------------------
%	BIBLIOGRAPHY
%-------------------------------------------------------------------------------

\bibliographystyle{plain}

\bibliography{refs}

%-------------------------------------------------------------------------------

\end{document}
