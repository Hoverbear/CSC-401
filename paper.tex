%%%%%%%%%%%%%%%%%%%%%%%%%%%%%%%%%%%%%%%%%
% Thin Sectioned Essay
% LaTeX Template
% Version 1.0 (3/8/13)
%
% This template has been downloaded from:
% http://www.LaTeXTemplates.com
%
% Original Author:
% Nicolas Diaz (nsdiaz@uc.cl) with extensive modifications by:
% Vel (vel@latextemplates.com)
%
% License:
% CC BY-NC-SA 3.0 (http://creativecommons.org/licenses/by-nc-sa/3.0/)
%
%%%%%%%%%%%%%%%%%%%%%%%%%%%%%%%%%%%%%%%%%

%-------------------------------------------------------------------------------
%	PACKAGES AND OTHER DOCUMENT CONFIGURATIONS
%-------------------------------------------------------------------------------

\documentclass[a4paper, 11pt]{article} % Font size (can be 10pt, 11pt or 12pt) and paper size (remove a4paper for US letter paper)

\usepackage[protrusion=true,expansion=true]{microtype} % Better typography
\usepackage{graphicx} % Required for including pictures
\usepackage{wrapfig} % Allows in-line images
\usepackage[none]{hyphenat} % Don't ever hypenate.
\usepackage{mathpazo} % Use the Palatino font
\usepackage{csquotes} % Block Quotes
\usepackage[T1]{fontenc} % Required for accented characters
\linespread{1.05} % Change line spacing here, Palatino benefits from a slight increase by default

\makeatletter
\renewcommand\@biblabel[1]{\textbf{#1.}} % Change the square brackets for each bibliography item from '[1]' to '1.'
\renewcommand{\@listI}{\itemsep=0pt} % Reduce the space between items in the itemize and enumerate environments and the bibliography

\renewcommand{\maketitle}{ % Customize the title - do not edit title and author name here, see the TITLE block below
\begin{flushright} % Right align
{\LARGE\@title} % Increase the font size of the title

\vspace{50pt} % Some vertical space between the title and author name

{\large\@author} % Author name
\\\@date % Date

\vspace{40pt} % Some vertical space between the author block and abstract
\end{flushright}
}

%-------------------------------------------------------------------------------
%	TITLE
%-------------------------------------------------------------------------------

\title{
    \textbf{The Impact of C-51 on Privacy}\\ % Title
    Toeing the delicate balance of privacy and security.
} % Subtitle

\author{\textsc{Hobden, Andrew and Wanger, Breck} % Author
\\{\textit{University of Victoria}}} % Institution

\date{\today} % Date

%-------------------------------------------------------------------------------

\begin{document}

\maketitle % Print the title section

%-------------------------------------------------------------------------------
%	ABSTRACT AND KEYWORDS
%-------------------------------------------------------------------------------

%\renewcommand{\abstractname}{Summary} % Uncomment to change the name of the abstract to something else

\begin{abstract}
The fabric of society relies on our ability to privately communicate with one another. Throughout our entire history as a species we have had the ability to do so, albeit limited by distance and latency. Technology has only in the past decades reached the speed and pervasiveness to enable truly private global communication, and now we must wrestle with the consequences.

Our digital infrastructure forms a substrate that connects us together, for better or worse. Technologies like the internet are almost overwhelmingly regarded as a betterment to our society. (TODO: CITE??) Contrary, as recent incidences of whistleblowing and conflict have shown, these technologies can be subverted and turned against us, being used to create terror. (TODO: more on data?)

When considering how to combat terrorism, it's important to remember the goal of those carrying out the actions. Creating terror, causing knee-jerk reactions, provoking radical changes. How can we say we are not giving into terror when we choose to sacrifice our liberty for our security? How can we say we are free when our most private communications are monitored to ensure we are not acting outside of the government's ideal view of society? How can we continue create good social change, such as universal suffrage was, in an atmosphere that includes suffocating surveilance?

In this paper, we explore how C-51 impacts the lives of normal citizens and activists.
\end{abstract}

\hspace*{3,6mm}\textit{Keywords:} Privacy, Canada, C-51 % Keywords

\vspace{30pt} % Some vertical space between the abstract and first section

%-------------------------------------------------------------------------------
%	ESSAY BODY
%-------------------------------------------------------------------------------

\section*{Background}
In this decade and the last in light of major threats to the security of the public from individuals and organizations deemed as terrorists by the government, legislation has been passed in an attempt to quell such risks. While these laws may be well intentioned they may also erode the freedoms of individuals and defiantly have created a chilling effect. There is an argument to be had for both sides which must be explored. The age old question that must be asked once again is: is it more important so safeguard a right to free speech or the general safety and security of the broader public.

On the heels of a domestic terrorism attack on the parliament of Canada, in 2015 the Conservative government introduced a new bill C-51. The bill was largely touted as an anti-terrorism bill but included many dangerous provisions which feature overly broad wording or presume strong watchdogs on Canada's spy and surveillance agencies.

During the parliamentary processes which granted this bill royal assent on June 18, 2015 there were widespread demonstrations and protests as public awareness about the bill rose, and approval of the bill fell. (TODO: CITEME) The arguments made by protesters were that...

%-------------------------------------------------------------------------------

\section*{Prior to C-51}

Cras gravida, est vel interdum euismod, tortor mi lobortis mi, quis adipiscing elit lacus ut orci. Phasellus nec fringilla nisi, ut vestibulum neque. Aenean non risus eu nunc accumsan condimentum at sed ipsum.
Aliquam fringilla non diam sed varius. Suspendisse tellus felis, hendrerit non bibendum ut, adipiscing vitae diam. Lorem ipsum dolor sit amet, consectetur adipiscing elit. Nulla lobortis purus eget nisl scelerisque, commodo rhoncus lacus porta. Vestibulum vitae turpis tincidunt, varius dolor in, dictum lectus. Aenean ac ornare augue, ac facilisis purus. Sed leo lorem, molestie sit amet fermentum id, suscipit ut sem. Vestibulum orci arcu, vehicula sed tortor id, ornare dapibus lorem. Praesent aliquet iaculis lacus nec fermentum. Morbi eleifend blandit dolor, pharetra hendrerit neque ornare vel. Nulla ornare, nisl eget imperdiet ornare, libero enim interdum mi, ut lobortis quam velit bibendum nibh.

%-------------------------------------------------------------------------------

\section*{Undermining the Security of Canada}
The act defines "activity that undermines the security of Canada" to mean any of the following activities:

\begin{enumerate}
    \item interference with the capability of the Government of Canada in relation to intelligence, defence, border operations, public safety, the administration of justice, diplomatic or consular relations, or the economic or financial stability of Canada;
    \item changing or unduly influencing a government in Canada by force or unlawful means;
    \item espionage, sabotage or covert foreign-influenced activities;
    \item terrorism;
    \item proliferation of nuclear, chemical, radiological or biological weapons;
    \item interference with critical infrastructure;
    \item interference with the global information infrastructure, as defined in section 273.61 of the National Defence Act;
    \item an activity that causes serious harm to a person or their property because of that person's association with Canada; and
    \item an activity that takes place in Canada and undermines the security of another state.
\end{enumerate}

After legitimate concerns cited the possibility of this being used to target non-violent domestic protests, such as the "Idle No More" movement or anti-pipeline protestors, a revision was added the language "For greater certainty, it does not include advocacy, protest, dissent and artistic expression." This clarification of language helped reduce the possibility for abuse, but does not eliminate it complete.

In order to further illustrate, consider where the line gets blurry, as in the following situation:

\begin{displayquote}
A First Nations band has chosen to block off an important shipping or logging road in protest of the constuction of a new oil pipeline. While this action certainly could be seen as undermining Canada's financial or economic stability, or endangering public safety, or even harming diplomatic relations if the pipeline was foreign owned. At the same time, this action is obviously a domestic protest and act of dissent.
\end{displayquote}

How does the bill characterize this event? Would the bill consider this "activity that undermines the security of Canada"? If not, at what point would it?

\section*{Information Sharing}
In the past, even in the past few decades, it woud have simply been impractical for a government to compile an accurate, detailed profile of a given person's life without considerable, targetted effort. In recent years this has changed, and now information sharing is considered a necessary feature of the government. \cite{total-awareness} Moreso, expansive technology companies such a Microsoft, Google, and Facebook have amassed a tremendous amount of data on citizens, marking such profiling as the norm, rather than the exception.

There is cause for concern with information sharing, even amongst government ministries. Despite providing the legalease to facilitate this sharing, between an open ended list of (at this time) 100 ministries and 17 bodies, the bill provides no steps towards ensuring the reliability of the provided data, leading to a risk of mis-identification and false evidence.

This very mistake of incorrect information was partially the cause of the case of Canadian citizen Maher Arar, when the RCMP provided information that the future Inquiry deemed "unfortunate, and to put it mildly, totally unacceptable." \cite{arar}

It is reasonable to assert that information sharing can be a useful feature for investigations into terrorism or otherwise defined acivities which undermine the security of Canada. There are questions on how to facilitate this while still respecting the privacy of normal citizens and those acting under provisions around protest, dissent, and advocacy.

Further, the question of how a person (or persons) wronged through this act may come to challenge such an action within the judicial system is left difficult. In many cases an affected person may not even come to know they are being affected until well after any damage has been done, if at all. A citizen may never know if they are ever flagged in a database or added to a watchlist.

\section*{A Starving Watchdog}
The Security Intelligence Review Committee, SIRC, is independent review body charged with overseeing the activities of CSIS. With CSIS's capabilities and responsibilies becoming significantly greater under the bill one would reasonably expect some level of capacity to be added to SIRC so that it may appropriately manage the new scope.

Unfortunately, this has not been the case. Over the past decade (under the then-incumbent Conversatives who introduced bill C-51) SIRC had seen it's budget remain flat, increasing from only \$2.9 to \$3 million, and it's employees shrink from 20 to 16. Worse, the incumbent government neglected to fill vacant seats within the committee over four years, leaving it with less members than indended. At the time, the current privacy commissoner warned against increasing surveillance powers without also increasing oversight. \cite{watchdog-starved}

\section*{C-51: For the Normal Citizen}
Lorem Ipstuff.

\section*{C-51: For the Activist}
Lorem Ipstuff.


\section*{Conclusion}
Fusce in nibh augue. Cum sociis natoque penatibus et magnis dis parturient montes, nascetur ridiculus mus. In dictum accumsan sapien, ut hendrerit nisi. Phasellus ut nulla mauris. Phasellus sagittis nec odio sed posuere. Vestibulum porttitor dolor quis suscipit bibendum. Mauris risus lectus, cursus vitae hendrerit posuere, congue ac est. Suspendisse commodo eu eros non cursus. Mauris ultrices venenatis dolor, sed aliquet odio tempor pellentesque. Duis ultricies, mauris id lobortis vulputate, tellus turpis eleifend elit, in gravida leo tortor ultricies est. Maecenas vitae ipsum at dui sodales condimentum a quis dui. Nam mi sapien, lobortis ac blandit eget, dignissim quis nunc.


%-------------------------------------------------------------------------------
%	BIBLIOGRAPHY
%-------------------------------------------------------------------------------

\bibliographystyle{unsrt}

\bibliography{refs}

%-------------------------------------------------------------------------------

\end{document}
